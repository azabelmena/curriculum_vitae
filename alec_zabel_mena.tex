%%%%%%%%%%%%%%%%%%%%%%%%%%%%%%%%%%%%%%%%%
% Medium Length Professional CV
% LaTeX Template
% Version 2.0 (8/5/13)
%
% This template has been downloaded from:
% http://www.LaTeXTemplates.com
%
% Original author:
% Rishi Shah
%
% Important note:
% This template requires the resume.cls file to be in the same directory as the
% .tex file. The resume.cls file provides the resume style used for structuring the
% document.
%
%%%%%%%%%%%%%%%%%%%%%%%%%%%%%%%%%%%%%%%%%

%----------------------------------------------------------------------------------------
%	PACKAGES AND OTHER DOCUMENT CONFIGURATIONS
%----------------------------------------------------------------------------------------

\documentclass{resume} % Use the custom resume.cls style

\usepackage[left=0.75in,top=0.6in,right=0.75in,bottom=0.6in]{geometry} % Document margins
\usepackage{multicol}

\usepackage{listings} % for placeholder images
\lstdefinestyle{mystyle}{
commentstyle=\color{codegreen},
keywordstyle=\color{magenta},
numberstyle=\tiny\color{codegray},
stringstyle=\color{codepurple},
basicstyle=\ttfamily,
breakatwhitespace=false,
breaklines=true,
captionpos=b,
keepspaces=true,
numbers=left,
numbersep=5pt,
showspaces=false,
showstringspaces=false,
showtabs=false,
tabsize=2
}
\lstset{style=mystyle}
\AtBeginEnvironment{tabular}{\ttfamily}

\newcommand{\tab}[1]{\hspace{.2667\textwidth}\rlap{#1}}
\newcommand{\itab}[1]{\hspace{0em}\rlap{#1}}
\name{Alec S. Zabel-Mena} % Your name
\address{alec.zabel@upr.edu \\ azabelmena@protonmail.ch} % Your address
\address{azabelmena.wordpress.com}

\begin{document}

\begin{rSection}{Education}
    \textbf{BS} \hspace*{10mm} Pure Mathematics, Minor in Cybersecurity
    \hfill{August 2016 - June 2022} \\
     \hspace*{17mm} University of Puerto Rico, R\'io Piedras

    \textbf{MS} \hspace*{10mm} Pure Mathematics
    \hfill{August 2022 - Present} \\
     \hspace*{17mm} University of Puerto Rico, R\'io Piedras \hspace*{32mm} In
     Progress
\end{rSection}

%\begin{rSection}{Career Objective}
    %To become an academic researcher and professor. Aiming to research both pure
    %and applied mathematics; specifically research in algebra, topolgy,
    %cryptography, coding theory, and other areas of discrete mathematics.
%\end{rSection}

%----------------------------------------------------------------------------------------
%	TECHNICAL STRENGTHS SECTION
%----------------------------------------------------------------------------------------

\begin{rSection}{Research}
    \textbf{An Efficient Implementation and Analysis of Zeta Functions of APN Curves
and Their Protograph LDPC Codes for Space Applications} \\
    University of Puerto Rico \hfill{August 2021 - Present}
    \begin{itemize}
        \item Researching APN functions and the classification of certain
            $2$-error correcting cyclic codes for use in cryptography and coding
            theory.

        \item Mentored by Professor Heeralal Janwa, Ph.D.
    \end{itemize}

    \textbf{The General Linear Group: Finding $2 \times 2$
    Representations of Finite Groups} \\
    University of Puerto Rico \hfill{April 2020}
    \begin{itemize}
        \item Final project for the second undergraduate seminar in mathematics
            (MATE3170). Research the general linear group on $2 \times 2$
            matrices and representations of well known groups using these
            matrices.

        \item Worked under the supervision of Professor Ra\'ul Figueroa, Ph.D.
    \end{itemize}

    \textbf{Matroid Theory}  \\
    University of Puerto Rico \hfill{November 2019}
    \begin{itemize}
        \item Final project for the first undergraduate seminar in mathematics
            (MATE3070). Gave a survey of the field of Matroid theory, and its
            applications in Graph theory, Topology, and Algorithm Design.

        \item Worked under the supervision of Professor Iv\'an Cardona, Ph.D.
    \end{itemize}

    \textbf{Algebraic Codes over Elliptic and Hermitian
    Curves} \\
    University of Puerto Rico \hfill{May 2019}
    \begin{itemize}
        \item Research paper, and final project of the Introduction to Coding
            Theory course. Studied and found algebraic geometric codes using
            elliptic and Hermitian curves for use in coding theory and
            cryptography.

        \item Worked under the supervision of Professor Heeralal Janwa, Ph.D.
    \end{itemize}
\end{rSection}

\begin{rSection}{Grants and Awards}
    \textbf{Puerto Rico Louis Stokes Alliance for Minority Participation} \\
    University of Puerto Rico \hfill{August 2021 - May 2022} \\

    \textbf{NASA PR Space Grant Fellowships and Scholarship Program (No.
    80NSSC20M0052)} \\
    University of Puerto Rico \hfill{August 2022 - May 2023}    \\

    \textbf{NASA PR Research Assistantship (No.
    80NSSC20M0052)} \\
    University of Puerto Rico \hfill{August 2023 - December 2023}    \\

    %\textbf{NASA PR Research Assistantship (No.
    %80NSSC20M0052)} \\
    %University of Puerto Rico \hfill{January 2024 - May 2024}    \\
\end{rSection}

%\pagebreak

\begin{rSection}{Research Interests}
    \begin{multicols}{2}
    \begin{itemize}
        \item Commutative Algebra

        \item Algebraic Geometry

        \item Finite Fields

        \item Topology and its use in other areas of mathematics

        \item Graph theory, Matroid theory, and Combinatorics

        \item Category Theory

        \item Algorithm Design for use in computer science and computational
            mathematics

        \item High Performance Computing

        \item Cybersecurity research and normalizing a culture that is privacy
            and security oriented

        \item Post-Quantum error correction
    \end{itemize}
    \end{multicols}
\end{rSection}

\begin{rSection}{Publications}
    \textbf{Works in Preparation} \\
    \hspace*{10mm} Janwa, H. Zabel-Mena, A. \textit{An Efficient Implementation and Analysis of Zeta Functions of APN Curves
and Their Protograph LDPC Codes for Space Applications}.
    \hspace*{10mm} unpublished.
\end{rSection}

\begin{rSection}{Presentations and Talks}
    \begin{itemize}
        %\item Zabel-Mena, Alec S., (2024, March). An Efficient Implementation and
            %Analysis Of Zeta Functions Of APN Curves and Their Protograph LDPC
            %Codes [minisymposium talk]. 39\textsuperscript{th} Seminario
            %Interuniversitario de Investicaci\'on en Ciencias Matem\'aticas,
            %San Juan, PR.

        \item Zabel-Mena, Alec S., (2023, November 30). An Efficient Implementation and
            Analysis Of Zeta Functions Of APN Curves and Their Protograph LDPC
            Codes [poster session]. 2023 Forward Research & Innovation Summit,
            San Juan, PR.

        \item Zabel-Mena, Alec S., (2022, April 09). An Efficient Implementation
            of Computing The Rational Points and Zeta-functions of Curves Associated
            with APN Monomials and Applications to Cyclic codes [minisymposium
            talk]. 55\textsuperscript{th} ACS Junior Technical Meeting, Humacao,
            PR.
    \end{itemize}
\end{rSection}

\begin{rSection}{Teaching and Mentoring}
    \textbf{Proyecto Tutor\'ias DECEP} \hfill{2020-2021} \\
    University of Puerto Rico, R\'io Piedras \\
    Mathematics tutor
    \begin{itemize}
        \item Conducted assessments to identify the educational needs of my
            students and developed individualized learning plans.

        \item Tasked with providing tutoring services to seven high school
            students for $10$ hours a week in the subjects of Algebra and
            Pre-Calculus.
    \end{itemize}

    \textbf{Self-Employed} \hfill{2019-2021} \\
    Mathematics tutor

    \begin{itemize}
        \item Evaluated students learning styles and provided appropriate
            techniques for maximizing understanding and minimizing frustration.

        \item Simplified math concepts while coaching students to think
            critically when problem solving; eventually introducing them to
            axiomatic systems within mathematics.

        \item Provided tutoring one-on-one to five highschool students in the
            subjects of Algebra, Pre-Calculus, and Geometry for two times a
            week, at two hours for each session.
    \end{itemize}
\end{rSection}

\begin{rSection}{Honors and Awards}
    Dean's List \hfill{2016 - 2021}
\end{rSection}

\begin{rSection}{Memberships and Affiliations}
    Asociaci\'on de Estudiantes de Ciencias de Computos (AECC) \hfill{2022 - Present} \\
    Asociaci\'on Estudiantil de Matem\'aticas (AeMAT) \hfill{2019-Present} \\
    \hspace*{17mm} Treasurer    \\
    Lembran\c{c}a Negra Capoeira \hfill{2016 - 2018} \\
    Senzala Capoeira \hfill{2018 - Present} \\
\end{rSection}

\begin{rSection}{Skills}
    \begin{itemize}
        \item \textbf{Languages}
            \begin{itemize}
                \item English

                \item Spanish

                \item Portuguese
            \end{itemize}

        \item \textbf{Software}
            \begin{itemize}
                \item Excel

                \item \LaTeX

                \item Knowledge of UNIX-like systems and the commandline to
                    streamline workflow and automate repetitive tasks

                \item Privacy and Security Oriented

                \item \lstinline{C} programming

                \item \lstinline{nix}
            \end{itemize}
    \end{itemize}
\end{rSection}

\end{document}
