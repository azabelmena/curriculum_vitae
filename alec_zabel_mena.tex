%%%%%%%%%%%%%%%%%%%%%%%%%%%%%%%%%%%%%%%%%
% Medium Length Professional CV
% LaTeX Template
% Version 2.0 (8/5/13)
%
% This template has been downloaded from:
% http://www.LaTeXTemplates.com
%
% Original author:
% Rishi Shah
%
% Important note:
% This template requires the resume.cls file to be in the same directory as the
% .tex file. The resume.cls file provides the resume style used for structuring the
% document.
%
%%%%%%%%%%%%%%%%%%%%%%%%%%%%%%%%%%%%%%%%%

%----------------------------------------------------------------------------------------
%	PACKAGES AND OTHER DOCUMENT CONFIGURATIONS
%----------------------------------------------------------------------------------------

\documentclass{resume} % Use the custom resume.cls style

\usepackage[left=0.75in,top=0.6in,right=0.75in,bottom=0.6in]{geometry} % Document margins
\usepackage{multicol}

\usepackage{verbatim}
\usepackage{listings}
\lstdefinestyle{mystyle}{
    backgroundcolor=\color{backcolour},
    commentstyle=\color{codegreen},
    keywordstyle=\color{magenta},
    numberstyle=\tiny\color{codegray},
    stringstyle=\color{codepurple},
    basicstyle=\ttfamily\footnotesize,
    breakatwhitespace=false,
    breaklines=true,
    captionpos=b,
    keepspaces=true,
    numbers=left,
    numbersep=5pt,
    showspaces=false,
    showstringspaces=false,
    showtabs=false,
    tabsize=2
}

\lstset{style=mystyle}

\newcommand{\tab}[1]{\hspace{.2667\textwidth}\rlap{#1}}
\newcommand{\itab}[1]{\hspace{0em}\rlap{#1}}
\name{Alec S. Zabel-Mena} % Your name
\address{alec.zabel@upr.edu \\ azabelmena@protonmail.ch} % Your address
%\address{azabelmena.wordpress.com}

\begin{document}
%----------------------------------------------------------------------------------------
%	EDUCATION SECTION
%----------------------------------------------------------------------------------------

\begin{rSection}{Education}
    \textbf{BS} \hspace*{10mm} Pure Mathematics \hfill{August 2016 - Present} \\
    \hspace*{17mm} Minor in Cybersecurity \\
     \hspace*{17mm} University of Puerto Rico, R\'io Piedras
\end{rSection}

%\begin{rSection}{Career Objective}
    %To become an academic researcher and professor. Aiming to research both pure
    %and applied mathematics; specifically research in algebra, topolgy,
    %cryptography, coding theory, and other areas of discrete mathematics.
%\end{rSection}

%----------------------------------------------------------------------------------------
%	TECHNICAL STRENGTHS SECTION
%----------------------------------------------------------------------------------------

\begin{rSection}{Research Experience}
    \textbf{Puerto Rico Louis Stokes Alliance for Minority Participation} \\
    University of Puerto Rico \hfill{August 2021 - Present}
    \begin{itemize}
        \item Researching APN functions and the classification of certain
            $2$-error correcting cyclic codes for use in cryptography and coding
            theory.

        \item Mentored by Professor Heeralal Janwa, Ph.D.
    \end{itemize}
\end{rSection}

\begin{rSection}{Research Work}
    \textbf{The General Linear Group: Finding $2 \times 2$
    Representations of Finite Groups} \\
    University of Puerto Rico \hfill{April, 2020}
    \begin{itemize}
        \item Final project for the second undergraduate seminar in mathematics
            (MATE3170). Research the general linear group on $2 \times 2$
            matrices and representations of well known groups using these
            matrices.

        \item Worked under the supervision of Professor Ra\'ul Figueroa, Ph.D.
    \end{itemize}

    \textbf{Matroid Theory}  \\
    University of Puerto Rico \hfill{November, 2019}
    \begin{itemize}
        \item Final project for the first undergraduate seminar in mathematics
            (MATE3070). Gave a survey of the field of Matroid theory, and its
            applications in Graph theory, Topology, and Algorithm Design.

        \item Worked under the supervision of Professor Iv\'an Cardona, Ph.D.
    \end{itemize}

    \textbf{Algebraic Codes over Elliptic and Hermitian
    Curves} \\
    University of Puerto Rico \hfill{May, 2019}
    \begin{itemize}
        \item Research paper, and final project of the Introduction to Coding
            Theory course. Studied and found algebraic geometric codes using
            elliptic and Hermitian curves for use in coding theory and
            cryptography.

        \item Worked under the supervision of Professor Heeralal Janwa, Ph.D.
    \end{itemize}
\end{rSection}

\begin{rSection}{Research Interests}
    \begin{multicols}{2}
    \begin{itemize}
        \item Group Theory

        \item Finite Fields

        \item Algebraic Geometry and the study of algebraic curves

        \item Topology and its use in other areas of mathematics

        \item Graph theory, Matroid theory, and Combinatorics

        \item Algorithm Design for use in computer science and computational
            mathematics

        \item Cybersecurity research and normalizing a culture that is privacy
            and security oriented

        \item Post-Quantum error correcting codes

        \item Post-Quantom implementations of cryptographic algorithms
    \end{itemize}
    \end{multicols}
\end{rSection}

\begin{rSection}{Teaching and Mentoring Experience}
    \textbf{Proyecto Tutor\'ias DECEP} \hfill{2020-2021} \\
    University of Puerto Rico, R\'io Piedras \\
    Mathematics tutor
    \begin{itemize}
        \item Conducted assessments to identify the educational needs of my
            students and developed individualized learning plans.

        \item Tasked with providing tutoring services to seven high school
            students for $10$ hours a week in the subjects of Algebra and
            Pre-Calculus.
    \end{itemize}

    \textbf{Self-Employed} \hfill{2019-2021} \\
    Mathematics tutor

    \begin{itemize}
        \item Evaluated students learning styles and provided appropriate
            techniques for maximizing understanding and minimizing frustration.

        \item Simplified math concepts while coaching students to think
            critically when problem solving; eventually introducing them to
            axiomatic systems within mathematics.

        \item Provided tutoring one-on-one to five highschool students in the
            subjects of Algebra, Pre-Calculus, and geometry for two times a
            week, at two hours for each session.
    \end{itemize}
\end{rSection}

\begin{rSection}{Publications.}
    \textbf{Works in Preparation} \\
    \hspace*{10mm} Janwa, H. Zabel-Mena A. \textit{APN Functions, and Classifying
    $2$-Error-Correcting Cyclic Codes}.
    \hspace*{10mm} unpublished.
    %\begin{itemize}
        %\item Researching APN functions and the classification of certain
            %$2$-error correcting cyclic codes for use in cryptography and coding
            %theory.

        %\item Research in progress, not currently published.

        %\item Under the supervision of Professor Heeralal janwa, Ph.D. for the
            %Puerto Rico Louis Stokes Alliance for Minority Participation
            %(PR-LSAMP).
    %\end{itemize}
\end{rSection}

\begin{rSection}{Honors and Awards}
    Dean's List \hfill{2016 - 2021}
\end{rSection}

\begin{rSection}{Memberships and Affiliations}
    Asociaci\'on de Estudiantes de Matem\'aticas (AeMAT) \hfill{2019-Present} \\
    Senzala Capoeira \hfill{2016 - 2020} \\
\end{rSection}

%\begin{rSection}{Presentations and Talks}
%\end{rSection}

\begin{rSection}{Professional Development}
    \textbf{Webinars}
    \begin{itemize}
        \item Going Down the Cyber Security Rabbit Hole \hfill{February 2022}
    \end{itemize}
    \textbf{Conferences attended as spectator}
    \begin{itemize}
        \item Interuniversity Seminar on Mathematical Sciences Research (SIDIM)
        \hfill{February 2022}
    \end{itemize}

\end{rSection}

\begin{rSection}{Skills}
    \begin{itemize}
        \item \textbf{Languages}
            \begin{itemize}
                \item English: Native

                \item Spanish: Native

                \item Portuguese: Basic (A2)
            \end{itemize}

        \item \textbf{Software}
            \begin{itemize}
                \item Excel

                \item \LaTeX

                \item Knowledge of UNIX-like systems and the commandline to
                    streamline workflow and automate repetitive tasks.

                \item Privacy and Security Oriented

                \item \lstinline{C/C++}
                    \begin{itemize}
                        \item[\circ] Used \lstinline{C++} to implement a
                            polynomial root finding algorithm in order to find
                            the number of points on a given rational surface.
                    \end{itemize}

                \item \lstinline{SAGE}
                    \begin{itemize}
                        \item[\circ] Used \lstinline{SAGE} to find elliptic
                            curves that attained the Hasse-Weil bound.
                    \end{itemize}
            \end{itemize}
    \end{itemize}
\end{rSection}

\end{document}
